% Start of code generated using Learning Module Creator

% <------------Linear First Order Begins Here------------>

	\section{Week 1 - Linear First Order}

	\subsection{Readings}
		Please read sections 1.2 and 2.1.

	\subsection{Words of Advice}
		\begin{itemize}
			\item Differential Equations is largely about identification of the problem.
			\item There'll be techniques to learn too, but knowing will be more than half the battle.
		\end{itemize}

	\subsection{Work}
		\begin{itemize}
			\item In section 1.2, please do problems 1, 2a-e and g, 3 eop, 4 eop, 5ad, 7.
			\item In section 2.1, please do problems 1-11 odd, 16, 17, 19, 20, 24, 31-33, 36, 47.
		\end{itemize}

	\clearpage

% <------------Linear First Order Ends Here------------>

% <------------Seperable Equations and ∃! Theorem Begins Here------------>

	\section{Week 2 - Seperable Equations and ∃! Theorem}

	\subsection{Readings}
		Please read sections 2.2 and 2.3.

	\subsection{Words of Advice}
		\begin{itemize}
			\item Separation of variables is a technique that we'll keep seeing at the end of problems.
			\item The existence and uniqueness theorem is much more crucial than it seems.
			\item If we didn't know a solution existed, we might just be wasting time trying to find it.
			\item If we weren't sure that solutions were unique, we couldn't trust the results from models.
			\item Don't confuse these with the idea that the solution we get from a model must match the reality of the situation. Knowing that model solutions are reasonable is a much bigger fish.
			\item Chaotic dynamics are still a thing even with the existence and uniqueness theorems.
			\item There are several Existence and Uniqueness theorems in the subject, this won't be our last.
		\end{itemize}

	\subsection{Work}
		\begin{itemize}
			\item In section 2.2, please do problems 1-6, 11-12, 17-23 odd, 29.
			\item In section 2.3, please do problems 1-13.
		\end{itemize}

	\clearpage

% <------------Seperable Equations and ∃! Theorem Ends Here------------>

% <------------Special Forms and Exactness Begins Here------------>

	\section{Week 3 - Special Forms and Exactness}

	\subsection{Readings}
		Please read sections 2.4 and 2.5.

	\subsection{Words of Advice}
		\begin{itemize}
			\item 'Homogeneous Non-Linear' and 'Homogeneous Linear' sound really similar but are pretty different.
			\item If you didn't take Calc 3 you'll need to understand partial derivatives to get exactness. (There's some videos for this in the Calc 3 playlist.)
		\end{itemize}

	\subsection{Work}
		\begin{itemize}
			\item In section 2.4, please do problems 1-4, 7-11, 15, 18, 23, 24, 28, 55, 57.
			\item In section 2.5, please do problems 1-21 odd, 29, 33.
		\end{itemize}

	\clearpage

% <------------Special Forms and Exactness Ends Here------------>

% <------------Integrating Factors and First Order Applications Begins Here------------>

	\section{Week 4 - Integrating Factors and First Order Applications}

	\subsection{Readings}
		Please read sections 2.6, 4.1, and 4.2.

	\subsection{Words of Advice}
		\begin{itemize}
			\item Integrating factors are a powerful technique, but not a time efficient one for general differential equations.
			\item The search for an integrating factor is a little bit akin to brute force password cracking: Will it eventually work? - Yes. Will the heatdeath of the universe occur beforehand? Quite possibly...
			\item Applications of ODEs are notoriously tricky, but nobody solves differential equations for funzies, they solve des because they'll understand the world a little better once they do.
			\item My advice for solving applications: Don't panic and write down some math, you don't have to know how to solve it as you write it down.
			\item Tank mixing problems come up in a ton of places in Engineering, get gud at them.
		\end{itemize}

	\subsection{Work}
		\begin{itemize}
			\item In section 2.6, please do problems 3-16 all.
			\item In section 4.1, please do problems 1-3, 8-12, 14-20.
			\item In section 4.2, please do problems 1-5.
		\end{itemize}

	\clearpage

% <------------Integrating Factors and First Order Applications Ends Here------------>

% <------------2olcchodes pronounced 'ta-wak-el-chodes' Begins Here------------>

	\section{Week 5 - 2olcchodes pronounced 'ta-wak-el-chodes'}

	\subsection{Readings}
		Please read sections 5.1 and 5.2.

	\subsection{Words of Advice}
		\begin{itemize}
			\item I didn't make this word up. Students did. I know it's silly, but it seems to cement the concept. :shrug:
			\item I'm also aware that the letters also aren't in that order. Complaints like this are why we can't have fun.
			\item 2olcchodes are not to be confused with 2olccnhodes (notice the additional 'n') pronounced 'ta-wak-lin-chodes', which are covered in 5.3.
			\item This section could also be called 'Springs and Pendulums', but I saved that for the applications.
		\end{itemize}

	\subsection{Work}
		\begin{itemize}
			\item In section 5.1, please do problems 1-4, 9, 10, 11, 13, 17, 24.
			\item In section 5.2, please do problems 1-22, 23-27 odd, 29, 33, 34.
		\end{itemize}

	\clearpage

% <------------2olcchodes pronounced 'ta-wak-el-chodes' Ends Here------------>

% <------------2olccnhodes pronounced 'ta-wak-lin-chodes' Begins Here------------>

	\section{Week 6 - 2olccnhodes pronounced 'ta-wak-lin-chodes'}

	\subsection{Readings}
		Please read sections 5.3, 5.7, and 5.4 is optional.

	\subsection{Words of Advice}
		\begin{itemize}
			\item The methods outlined in the homework for 5.3 are the basis of a technique called Method of Undetermined Coefficients.
			\item MUDCs is an effective method for solving a great many 2olccnhodes, but I personally prefer Variation of Parameters.
			\item If you'd like to make your own determination about this, MUDCs is covered in 5.4 and 5.5. You may well run across it in Engineering classes.
			\item VOP is the prefered technique in my mind because it lends itself to higher order situations and is a lot easier to remember because MUDCs invlives lots of cases...
		\end{itemize}

	\subsection{Work}
		\begin{itemize}
			\item In section 5.3, please do problems 1-4, 16, 17, 39.
			\item In section 5.7, please do problems 1, 3, 7, 8, 17, 18, 23, 25.
		\end{itemize}

	\clearpage

% <------------2olccnhodes pronounced 'ta-wak-lin-chodes' Ends Here------------>

% <------------Springs and Pendulums Begins Here------------>

	\section{Week 7 - Springs and Pendulums}

	\subsection{Readings}
		Please read sections 6.1 and 6.2.

	\subsection{Work}
		\begin{itemize}
			\item In section 6.1, please do problems 2, 4, 6, 8, 13.
			\item In section 6.2, please do problems 6, 9, 10, 15, 20.
		\end{itemize}

	\clearpage

% <------------Springs and Pendulums Ends Here------------>

% <------------Midterm and Series Solutions Begins Here------------>

	\section{Week 8 - Midterm and Series Solutions}

	\subsection{Readings}
		Please read sections 7.1 and 7.2.

	\subsection{Words of Advice}
		\begin{itemize}
			\item In my mind, chapter 7 is a bit too complicated for this class, but it's an important concept, so we'll dip our toes in.
			\item The idea of a power series solution to an equation is a pretty common basis for coding algorithms, but it's really hard to do by hand.
			\item If you're headed into applied math, this chapter would be a really good after the course self study.
		\end{itemize}

	\subsection{Work}
		\begin{itemize}
			\item In section 7.1, please do problems 1abce, 20, 21.
			\item In section 7.2, please do problems 1-3.
		\end{itemize}

	\clearpage

% <------------Midterm and Series Solutions Ends Here------------>

% <------------Laplace Transforms Begins Here------------>

	\section{Week 9 - Laplace Transforms}

	\subsection{Readings}
		Please read sections 8.1 and 8.2.

	\subsection{Words of Advice}
		\begin{itemize}
			\item The Laplace transform doesn't necessarily make a problem easier, but it does shift the difficulty to algebraic techniques, which are a bit simpler.
			\item The Laplace Transform is also the only tool we have to deal with disconinuous forces, which are encoutered everyday.
			\item You literally cannot go to the grocery store without encountering a discontinuous forcing term.
		\end{itemize}

	\subsection{Work}
		\begin{itemize}
			\item In section 8.1, please do problems 1 and 2eop, 5bc, 6, 15, 16.
			\item In section 8.2, please do problems 1 all, 2-4 eop, and 6-8 eop.
		\end{itemize}

	\clearpage

% <------------Laplace Transforms Ends Here------------>

% <------------IVPs and Unit Steps Begins Here------------>

	\section{Week 10 - IVPs and Unit Steps}

	\subsection{Readings}
		Please read sections 8.3 and 8.4.

	\subsection{Words of Advice}
		\begin{itemize}
			\item The whole point of this technique is to solve IVPs.
			\item The Laplace transform relies on you having those initial values in order to solve des.
			\item Unit step functions model switched behaviors.
			\item If you've never seen a switch before, it's a stick that has two positions, usually one makes a thing off and the other makes the thing on.
			\item You can think of a place you've seen a switch before right?
		\end{itemize}

	\subsection{Work}
		\begin{itemize}
			\item In section 8.3, please do problems 3-36 by 6s.
			\item In section 8.4, please do problems 7-14 odds, 19, 20, 29.
		\end{itemize}

	\clearpage

% <------------IVPs and Unit Steps Ends Here------------>

% <------------Piecewise Forcing and Dirac Deltas Begins Here------------>

	\section{Week 11 - Piecewise Forcing and Dirac Deltas}

	\subsection{Readings}
		Please read sections 8.5 and 8.7.

	\subsection{Words of Advice}
		\begin{itemize}
			\item If you own a hammer, this is a model you need.
			\item Given the prevalence of procussive problem solving in engineering fields, Dirac's Delta function is underappreciated.
		\end{itemize}

	\subsection{Work}
		\begin{itemize}
			\item In section 8.5, please do problems 1-5, 7.
			\item In section 8.7, please do problems 1-7 odd, 20, 24.
		\end{itemize}

	\clearpage

% <------------Piecewise Forcing and Dirac Deltas Ends Here------------>

% <------------Systems of Differential Equations Begins Here------------>

	\section{Week 12 - Systems of Differential Equations}

	\subsection{Readings}
		Please read sections 10.1 and 10.2.

	\subsection{Words of Advice}
		\begin{itemize}
			\item Systems of des are far and away the most common case. It's very rare for a problem to involve a single value changing in time.
			\item A linear algebra class is a good idea.
			\item Interconnected tanks are really really cool.
			\item Markov processes are another common model that feel very similar to interconnected tanks.
		\end{itemize}

	\subsection{Work}
		\begin{itemize}
			\item In section 10.1, please do problems 1, 2, 5abc.
			\item In section 10.2, please do problems 1 bd, 2 bd, 3a, 4a, 5bd, 8eop (start with b).
		\end{itemize}

	\clearpage

% <------------Systems of Differential Equations Ends Here------------>

% <------------Socchodes pronounced 'Sock-Hodes' Begins Here------------>

	\section{Week 13 - Socchodes pronounced 'Sock-Hodes'}

	\subsection{Readings}
		Please read sections 10.3 and 10.4.

	\subsection{Words of Advice}
		\begin{itemize}
			\item I bet you wish you'd signed up for linear... :)
			\item Eigenpairs are super cool and generally fairly hard to find.
			\item If you're in Linear I bet you can see why we'd want a Diagnalization right about now...
			\item After you've calculated a few eignepairs 
		\end{itemize}

	\subsection{Work}
		\begin{itemize}
			\item In section 10.3, please do problems 1, 7.
			\item In section 10.4, please do problems 1-8 (use a computer for #8s eigenvalues), 16-19.
		\end{itemize}

	\clearpage

% <------------Socchodes pronounced 'Sock-Hodes' Ends Here------------>

% <------------More Socchodes Begins Here------------>

	\section{Week 14 - More Socchodes}

	\subsection{Readings}
		Please read sections 10.5 and 10.6.

	\subsection{Words of Advice}
		\begin{itemize}
			\item There's a real rabbit hole to fall down here about the algebraic and geometric multiplicity of eigenvalues.
			\item We're going to avoid it for the most part, but I'd feel bad not pointing it out.
		\end{itemize}

	\subsection{Work}
		\begin{itemize}
			\item In section 10.5, please do problems 1-4, 13, 14.
			\item In section 10.6, please do problems 1-3, 9, 17, 18.
		\end{itemize}

	\clearpage

% <------------More Socchodes Ends Here------------>

% <------------Review for Final Begins Here------------>

	\section{Week 15 - Review for Final}

	\subsection{Words of Advice}
		\begin{itemize}
			\item You got this.
			\item You've got gud.
			\item Seriously though, good work for making it to the end. I really appreciate all the effort.
		\end{itemize}

	\clearpage

% <------------Review for Final Ends Here------------>

% End of code generated using Learning Module Creator

