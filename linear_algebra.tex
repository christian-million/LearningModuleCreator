% Start of code generated using Learning Module Creator

% <------------Introduction to Vectors Begins Here------------>

	\section{Week 1 - Introduction to Vectors}

	\subsection{Readings}
		Please read sections 1.1, 1.2, and 1.3.

	\subsection{Words of Advice}
		\begin{itemize}
			\item The picture for problem 1.1 23 is on the previous page.
			\item The whole course is about linear combinations and vectors, don't slack off here.
			\item It might be a legit idea to do all the problems in this seciton.
			\item The other book chooses to delay discussions of dot products until very late. I personally believe this is a mistake.
			\item Do the worked out examples and try not to look at the solution key until you've actually worked on a problem for a bit.
			\item Answers don't count for much, explanations are where all the points are, so get used to writing out your thoughts for each step.
			\item You should be explaining at a level your fellows would understand.
			\item 1.2 numbers 26 and 27 are meant to be a single problem.
			\item Problems 4 and 5 in 1.3 are the heart of the subject. Problem 4 is the 'Column multiplication picture' and problem 5 is the 'row multiplication picture'.
			\item For problem 7 in 1.3, the xs are unknown but fixed values, first assume that they aren't 0. After you're done, consider the possiblity that the xs in  are 0, is the conclusion wrong? 
			\item Hints for problems 4 and 5 from 1.3 can be found by reading 2.1.
		\end{itemize}

	\subsection{Work}
		\begin{itemize}
			\item In section 1.1, please do problems 1, 3, 4, 6, 9, 11, 13, 18, 23, 26-29, and 31.
			\item In section 1.2, please do problems 1, 2, 6, 10, 13, 16, 20, 26/27 (these are a single problem), 29, 30 .
			\item In section 1.3, please do problems 1, 2-6, 7, 14.
		\end{itemize}

	\clearpage

% <------------Introduction to Vectors Ends Here------------>

% <------------Systems and Elimination Begins Here------------>

	\section{Week 2 - Systems and Elimination}

	\subsection{Readings}
		Please read sections 2.1, 2.2, and 2.3.

	\subsection{Words of Advice}
		\begin{itemize}
			\item The two pictures (column and row) in section 2.1 are the first step towards understanding the picture on the front of the book.
			\item They're two sides of the same thought, but seem very different.
			\item If you solve something using one of them consider trying it with the other. It won't always work, but it is good to consider.
			\item A matrix is a function. An $m\times n$ matrix is a function from the n dimensional real numbers to the m dimensional real numbers.
			\item Elimination matricies are fairly tough, I'm keeping the assingments in 2.2 and 2.3 a little light.
			\item For number 20 in 2.2, to see the three planes described at the beginning, make a triangular tube out of a piece of paper.
			\item For number 32 in 2.2, you should use some ..., don't fill a page with numbers.
			\item The answer key answer for number 32 part d is describing the row and column spaces for the second matrix they describe in part c, not the first one.
			\item Problem 28 in section 2.3 is fairly tough to do without a hint, but it's really simple with one. For reference the 'associative law' is on page 61.
		\end{itemize}

	\subsection{Work}
		\begin{itemize}
			\item In section 2.1, please do problems 1, 4-13, 15-17, 19, 20, 22, 26, 28, 31.
			\item In section 2.2, please do problems 3, 4-6, 11, 13, 20, 24, 32.
			\item In section 2.3, please do problems 1-3, 6, 7, 12, 13, 16, 17, 19, 24, 27, 28.
		\end{itemize}

	\clearpage

% <------------Systems and Elimination Ends Here------------>

% <------------Matrix Operations and Inverses Begins Here------------>

	\section{Week 3 - Matrix Operations and Inverses}

	\subsection{Readings}
		Please read sections 2.4 and 2.5.

	\subsection{Words of Advice}
		\begin{itemize}
			\item It is incredibly important that you know that matricies don't commute.
			\item $AB$ is NOT generally $BA$.
			\item Even if $AB$ is $BA$, this doesn't mean that $A$ or $B$ commute with pretty much anything else.
			\item Matrix multiplication is function composition, which is why it's not commutative.
			\item Exponents (repeated application of the matrix) have the usual rules.
			\item Matricies are 'linear', meaning (for vectors $x$ and $y$, matrix $A$ and scalar $c$) $A(x+y) = Ax+Ay$ and $Acx = cAx$
			\item The worked example 2.4B illustrates some important concepts.
			\item Functions are only invertible if they're one to one.
			\item The good old horizontal line test is checking this. Specifically, we need to check if there are two inputs that give us the same output.
			\item For a regular function $f(x)$, this looks like a pair of $a$ and $b$ with $a != b$ and $f(a) = f(b)$.
			\item In the special case of a matrix, this would look like $x$ and $y$ with $x != y$ and $Ax = Ay$. Because we know $A$ is linear, we can rearrange this to $Ax-Ay=0$, which can be further rearranged into $A(x - y) =0$.
			\item Ultimately, a matrix fails to be invertible if there is a nonzero vector $x$ with $Ax=0$. This will keep coming back up.
			\item The determinant gets a casual mention in this section, some approaches to this topic focus almost exclusively on the determinant. Those appraches are less effective in my opinion, but they do exist.
			\item Worked example 2.5B is a pretty crucial concept, expect to use this repeatedly.
			\item Order Matters. A supervising professor of mine wrote a precalculus book that had this phrase bolded every other paragraph or so. We made fun of it a bit, but it turns out it's just super solid life advice.
		\end{itemize}

	\subsection{Work}
		\begin{itemize}
			\item In section 2.4, please do problems 1-3, 5-7, 11, 15, 22, 23, 32, 34.
			\item In section 2.5, please do problems 1, 2, 7-9, 11, 12, 15, 18, 22, 29, 30, 39.
		\end{itemize}

	\clearpage

% <------------Matrix Operations and Inverses Ends Here------------>

% <------------LU Factorization and Transposes Begins Here------------>

	\section{Week 4 - LU Factorization and Transposes}

	\subsection{Readings}
		Please read sections 2.5 and 2.6.

	\subsection{Words of Advice}
		\begin{itemize}
			\item Factorization seems fairly silly, but as we develop this area, we'll find it increasingly useful.
			\item Linear algebra calculations are crucial to a large number of computer systems. Keeping those computer systems efficient is a huge task and factorizations often let us cut large amounts of calculations.
			\item Those of you who took Calc 3 will recall how totally crucial dot products were. It will come as no surprise that dot products are crucial here.
			\item Transpose matricies let us capture some of that utility for a more general application.
			\item Transpose matricies will continue to be seen throughout the class. In fact, they're crucial to understanding the cover art.
			\item The transpose of a matrix $A$ can also be thought of as $A$ flipped across the main diagonal.
			\item A somewhat useful notational trick is to recognize $(Ax)\cdot y = (Ax)^Ty = x^TA^Ty = x\cdot A^Ty$. 
			\item There's some discussion of this idea on page 121 with respect to derivatives.
		\end{itemize}

	\subsection{Work}
		\begin{itemize}
			\item In section 2.6, please do problems 1, 2, 5-8, 12, 13.
			\item In section 2.7, please do problems 1-4, 7bcd, 9, 10, 13a, 16, 17a, 28, 30, 37, 39.
		\end{itemize}

	\clearpage

% <------------LU Factorization and Transposes Ends Here------------>

% <------------Midterm 1 and Vector Spaces Begins Here------------>

	\section{Week 5 - Midterm 1 and Vector Spaces}

	\subsection{Readings}
		Please read sections 3.1.

	\subsection{Words of Advice}
		\begin{itemize}
			\item Fully explain all of your answers to the test. There's no credit for solutions without explanation.
			\item Explanations should cite facts from the book by page number and location. For example, 'top of page 123' or 'eqn (2) on page 123'.
			\item Justification should be provided for each step.
			\item You get credit for sound mathematical reasoning and careful explanation.
			\item Your explanations should be aimed at your peers level, not at me or Prof. Strang.
			\item Tests are assumed to be open book and notes, but are individual assessments.
			\item Please don't collaborate on tests.
			\item After you've finished your test, take on section 3.1.
			\item Section 3.1 is crucial, be sure to devote adequate time to it.
			\item A real vector space is a set of vectors and a pair of operations 'vector addition' and 'scalar multiplication' that is closed (see top of p125) under both operations and satisfys items (1) - (8) on p130.
			\item Vectors in these vector spaces do not need to look like the usual vectors in $\mathbb{R^n}$.
			\item Vector addition and scalar multiplication are not always definied in the usual way.
			\item The 'zero vector' need not actually be 0. See problem 3.
			\item Vector multiplication is rarely defined. (Dot and cross product aren't really multiplication.) Vector division is almost never defined, with the notable exceptions of $\mathbb{R^1}$ and $\mathbb{C}$.
			\item If the homework in section 3.1 seems easy, you're either doing it wrong or you're ready for higher math.
			\item The column space of a matrix is a critical concept moving forward, don't neglect it.
			\item There is at least one column space in the cover art.
			\item The symbol $\cup$ is called the 'union'. $S\cup T$ is the set of all things that are in $S$ or in $T$. For example $\{1,2,3\} \cup \{2,4,6 \} = \{1,2,3,4,6 \}$.
			\item The symbol $\cap$ is called the 'intersection'. $S\cap T$ is the set of all things that are in both $S$ and $T$. For example $\{1,2,3\} \cap \{2,4,6 \} = \{2 \}$.
			\item There will be several important subspaces associated to a matrix, all of which are the span of a set of vectors. The concept of span becomes more important as we proceed.
			\item 'Subspaces are flat.' Consider this statement carefully and see if you can justify it.
		\end{itemize}

	\subsection{Work}
		\begin{itemize}
			\item In section 3.1, please do problems 1-11, 14, 15, 17, 19, 22-24, 30.
		\end{itemize}

	\clearpage

% <------------Midterm 1 and Vector Spaces Ends Here------------>

% <------------Nullspaces and Complete Solutions Begins Here------------>

	\section{Week 6 - Nullspaces and Complete Solutions}

	\subsection{Readings}
		Please read sections 3.2 and 3.3.

	\subsection{Words of Advice}
		\begin{itemize}
			\item The rank of a matrix is very important, it's the dimension of the image (also called the range).
			\item The lines on the top of p140 shouldn't be ignored, they're going to be used alot.
			\item We kinda lied to you about domains and ranges. Let's clear that up a little bit. The domain of a function is the set of valid inputs, the range is the set of all outputs, and a third set called the 'codomain' is the space containing the range.
			\item For example, consider the matrix $A = \begin{bmatrix} 1 & 0 & 0\\ 0 & 0 &0 \end{bmatrix}$. This function has a domain of $\mathbb{R^3}$, a range of $\text{span}\begin{bmatrix} 1\\0 \end{bmatrix}$, and a codomain of $\mathbb{R^2}$.
			\item The Counting Theorem on p140 and in problem 4 is crucial to the sizes of the boxes on the cover art.
			\item Problems 48-50 are important, but conceptually hard and the solution manual explanations are a bit short in my opinion. Please make your explanations more complete. For problem 49, you may use that $rank(A) = rank(A^T)$ for any matrix $A$.
			\item It's sometimes said that mathematicians really only use three numbers, 0, 1, and $\infty$. This section illustrates the truth of that.[:)]
			\item All of this section basically rests on the idea that $Ax= b$ is equivalent to $A(x_p+x_s) = Ax_p +Ax_s = b + 0$.
			\item Now you can probably explain why the cover art contains $Ay = b$, $y=x+z$, $Ax= b$, and $Az=0$. If you'd like to check your understanding, please explain it on your coversheet for this module.
		\end{itemize}

	\subsection{Work}
		\begin{itemize}
			\item In section 3.2, please do problems 1, 2, 4, 5, 8, 9, 12, 16, 20-22, 24a, 29, 33, 39, 41, 48-50..
			\item In section 3.3, please do problems 3, 4, 6, 10, 11, 12, 13, 22, 24, 31, 34ac.
		\end{itemize}

	\clearpage

% <------------Nullspaces and Complete Solutions Ends Here------------>

% <------------Dimensions Begins Here------------>

	\section{Week 7 - Dimensions}

	\subsection{Readings}
		Please read sections 3.4 and 3.5.

	\subsection{Words of Advice}
		\begin{itemize}
			\item Other dimensions, not quite what sci-fi makes them out to be eh?
			\item I'd prefer if the author refered to sets of vectors being linearly independant rather than sequences of vectors because the ordering is not important.
			\item No set containing the zero vector can be linearly independent.
			\item The geometry of independence is a good intuition to have.
			\item A space or subspace will have a great many possible basis sets to choose from.
			\item Function spaces are super cool.
			\item The zero vector space (only the zero vector) is a little confusing, because it is dimension 0 and its basis is the empty set (the set of no vectors). See top of p. 172.
			\item Spans of sets are subspaces and subspaces are flat.
			\item For problem 3.4 45, you'll probably want to consider a basis for each V and W.
			\item 3.5, the big picture! Here we go, time to think about those rectangles in the cover art!
			\item Remember problems 48-50 in the nullspace section? The results are summarized again in Wroked Example 3.5B.
			\item Personally, I tend to say 'the nullspace of A transpose' rather than 'left nullspace of A'. Perhaps this is a personality deficit, but I prefer to think of $A$ as a map from left to right of the Big Picture and $A^T$ as a map from right to left.
			\item The Fundamental Theorem and the Big Picture in this section are the foundations of linear algebra. Everything before this was roughly so that you could understand this, and everything after it is basically a consequence of it. Frankly, this section is the subject of Linear Algebra.
			\item All the mysteries of the cover art are not yet solved, but this should clarify a great deal of it.
			\item There is an element that could be added to the cover art involving the dimensions of the spaces, what is it?
		\end{itemize}

	\subsection{Work}
		\begin{itemize}
			\item In section 3.4, please do problems 1, 2, 5, 8, 9, 11, 15, 16, 24, 26, 31, 32, 35, 38, 45.
			\item In section 3.5, please do problems 1, 2, 4, 5, 6, 10, 11, 13ab, 16, 24, 29.
		\end{itemize}

	\clearpage

% <------------Dimensions Ends Here------------>

% <------------Orthogonality and Projections Begins Here------------>

	\section{Week 8 - Orthogonality and Projections}

	\subsection{Readings}
		Please read sections 4.1 and 4.2.

	\subsection{Words of Advice}
		\begin{itemize}
			\item Orthogonal complements, super cool and also maybe you realized this a long time ago. Some folks do, usually because they notice that a vector in the nullspace must be perpendicular to the rows and the two spaces dimensions add to the domains.
			\item More Big Picture! On p 198 you can see a possible subtle change to the cover art. See how the Nullspaces are different sizes, but the row and col spaces are the same size? This is what I was asking for in the last module.
			\item The decomposition on page 199 ($x = x_r + x_n$) is somewhat surprising and also cool. It's the $y = x+z$ from the cover art.
			\item Worked Example 4.1A is a good understanding check.
			\item Problems 6 and 7 in 4.1 are not required, but if you're interested in applied math, the Fredholm Alternative makes several appearances.
			\item Problem 4.1 9 will come up again in the next couple sections, keep your eye on it.
			\item Students comfortable with calc 3 may prefer to phrase one dimensional projections in terms of dot products, but rephrasing in terms of transposes will let us lift this idea to matricies in general.
			\item Projection matricies onto the axises or coordinate planes are a useful tool in a number of contexts. The vast majority of applications of matricies outside of Linear Algebra utilize either projection or rotation matricies.
			\item To make the projection matrix on page 208 sound as fancy as possible, you can refer to it as the outside-in product of a with itself. This isn't serious, but you might notice that the top is the outer product of a with a and the bottom is the inner product of a with a.
			\item Be careful, order matters.
			\item $A^TA$ is invertible if A has linearly indep columns, but in this context A is rectanglular, therefore not invertible.
			\item $P^2 = P$ is sometimes taken to be the definition of a projection, you might convince yourself that this makes sense. It'll be the focus of some problems.
			\item Notice that there's a picture for problem 5.
			\item If you have extra time, question 34 is cool and doesn't have an answer in the solutions. Perhaps you can prove this.
		\end{itemize}

	\subsection{Work}
		\begin{itemize}
			\item In section 4.1, please do problems 1-4, 5a, 9-11, 13, 16(See page 195 for (2)), .
			\item In section 4.2, please do problems 1, 3, 5, 8, 11-13, 17, 18, 21, 22, 23, 24, 25, 32, 34 optional.
		\end{itemize}

	\clearpage

% <------------Orthogonality and Projections Ends Here------------>

% <------------Least Squares and Gram-Schmidt Begins Here------------>

	\section{Week 9 - Least Squares and Gram-Schmidt}

	\subsection{Readings}
		Please read sections 4.3 and 4.4.

	\subsection{Words of Advice}
		\begin{itemize}
			\item Least Squares regression is probably the most common statistics tool.
			\item The exceptionally cool thing about this section is that we learn that least squares lines aren't really any easier (in theory) than any other least squares polynomial fit.
			\item All of this is a little contrived by hand. Programming a computer to do this quickly is super easy.
			\item Our usual basis is an orthonormal basis, and the desire to go back to it or think in terms of it should motivate us to find more othornormal basises.
			\item The rotation matricies mentioned in Example 1 are a very common tool.
			\item Notice that orthogonal matricies are square.
			\item Careful, there are permutation matricies in this section, don't get them mixed up with projections.
			\item Squaring problem 4b and problem 10 in 4.4 seems like a tall order. Why are both possible? Read carefully.
		\end{itemize}

	\subsection{Work}
		\begin{itemize}
			\item In section 4.3, please do problems 1-3, 5, 9, 10, 17, 20, 21, 25.
			\item In section 4.4, please do problems 1, 2, 4bc, 5, 10, 11, 13, 17, 18, 20, 21, 22.
		\end{itemize}

	\clearpage

% <------------Least Squares and Gram-Schmidt Ends Here------------>

% <------------Test 2 and Determinants Begins Here------------>

	\section{Week 10 - Test 2 and Determinants}

	\subsection{Readings}
		Please read sections 5.1.

	\subsection{Words of Advice}
		\begin{itemize}
			\item Much of the advice I gave for the first test applies here, but I'm writing this before I've seen your first exams, so please bear in mind any advice I gave you upon the retrun of those tests.
			\item Fully explain all of your answers to the test. There's no credit for solutions without explanation.
			\item Explanations should cite facts from the book by page number and location. For example, 'top of page 123' or 'eqn (2) on page 123'.
			\item Justification should be provided for each step.
			\item You get credit for sound mathematical reasoning and careful explanation.
			\item Your explanations should be aimed at your peers level, not at me or Prof. Strang.
			\item Tests are assumed to be open book and notes, but are individual assessments.
			\item Please don't collaborate on tests.
			\item After you've finished your test, take on section 5.1.
			\item Section 5.1 is crucial, be sure to devote adequate time to it.
			\item IMPORTANT: Determinants are only defined for square matricies.
			\item Many other texts devote a large portion of the start of this course to calculation of the determinant. Personally I find that approach to be boring, because I find calculation by hand to be boring. The determinant is so useful that it can't be ignored, so here we are.
			\item A computer or a calculator is the right tool for finding a determinant. We'll do some by hand here, but it's not ideal.
			\item In chapter 6 we'll end up doing more than a little calculation of determinants by hand. In realiztic applications, these are usually calculated using a computer.
			\item The problems involving $A-\lambda I$ are foreshadowing for chapter 6.
		\end{itemize}

	\subsection{Work}
		\begin{itemize}
			\item In section 5.1, please do problems 1-4, 6, 7 (only the first Q), 8,  10, 22, 23, 28.
		\end{itemize}

	\clearpage

% <------------Test 2 and Determinants Ends Here------------>

% <------------Determinant Formulas Begins Here------------>

	\section{Week 11 - Determinant Formulas}

	\subsection{Readings}
		Please read sections 5.2 and 5.3.

	\subsection{Words of Advice}
		\begin{itemize}
			\item There are a ton of scary ass formulas in here. The cofactor formula is the one that gets the most use in real life.
			\item Most of these calculations are done recursively on computers. Recursion is when a computer program starts another copy of itself, usually to do a smaller calculation and report back.
			\item Notice that the cofactors switch signs.
			\item Cofactors are especially useful when there are lots of zeros.
			\item Problem 5.2 15 and 16 are a bit challenging to see because you have to go just a little deeper than you'd expect.
			\item Cramer's Rule lends itself to computer calculation of matrix inverses because it involves no possibility of required row swapping.
			\item It's straightforward execution is what makes it valuable, sadly it's not particularly easy to calculate unless you're a computer.
			\item The area interpretations of the determinant should come as no surprise to those who took calc 3.
			\item For those not having taken calc 3, the cross product is an important tool there for finding the areas of surfaces.
			\item Example 6 is another perspective on double integration in polar coordinates. Anybody care to extrapolate to cylindrical or spherical?
			\item You can skip 27 and 28 if you didn't take calc 3.
		\end{itemize}

	\subsection{Work}
		\begin{itemize}
			\item In section 5.2, please do problems 1, 3, 5, 7, 15, 16, 28, 34.
			\item In section 5.3, please do problems 1a, 7, 11, 12, 16, 19, 20, 26, 27, 28, 31, 32, 36, 38.
		\end{itemize}

	\clearpage

% <------------Determinant Formulas Ends Here------------>

% <------------Eigenvalues Begins Here------------>

	\section{Week 12 - Eigenvalues}

	\subsection{Readings}
		Please read sections 6.1 and 6.2.

	\subsection{Words of Advice}
		\begin{itemize}
			\item Eigenvectors are directions where the matrix $A$ acts like scalar multiplication, the scalar it acts like is the eigenvalue.
			\item I tend to refer to an eigenvector and its eigenvalue as an 'eigenpair'. I intentionally said 'eigenvector and its eigenvalue', because one eigenvalue may have a couple eigenvectors.
			\item Eigenpairs have seemingly endless uses. In an upcoming section we'll see how they can solve differential equations.
			\item Eigenpairs also appear in computer graphics, population modeling, dynamics, and a myriad of other applicaitons.
			\item A Big Picture similar to the front of the book could be created for a two by two matrix with distinct eigenpairs.
			\item Diagnalizing matricies is a major step in reducing computation time. Computation time of this form is the major cost associated to Machine Learning Algorithms, so reducing it is a major payoff.
			\item The optional problems in 6.2 (8 and 10) deal with the Fibonacci numbers, You need not do problem 8 to do problem 10. You can do it simply with the recursive definition found in problem 16 on page 267.
			\item Problem 6.2 26 is a callback to the Big Picture.
			\item Problem 34 is really really cool. It couples a geometrically obvious fact with a neat proof in the complex numbers. I wonder if there's a proof available along a path of medium difficulty.
			\item There is a great deal more to eigenvectors to explore including complex values and vectors, generalized eigenvectors, and some serious attempts to get eigenvalues with algebraic or geometric multiplicity greater than one to play nicely. Entire courses exist in these areas.
		\end{itemize}

	\subsection{Work}
		\begin{itemize}
			\item In section 6.1, please do problems 1, 2, 5, 6, 9, 10, 12, 14, 21, 25, 32, 37, 38.
			\item In section 6.2, please do problems 1-4, 6, 7, 8 optional, 10 optional, 11, 14, 15, 26, 34.
		\end{itemize}

	\clearpage

% <------------Eigenvalues Ends Here------------>

% <------------Differential Equations and Symmetric Matricies Begins Here------------>

	\section{Week 13 - Differential Equations and Symmetric Matricies}

	\subsection{Readings}
		Please read sections 6.3 and 6.4.

	\subsection{Words of Advice}
		\begin{itemize}
			\item This isn't an entire course in differential equations, but this useful tool is such an easy pickup from where we are that it can't be ignored.
			\item The definition of the word exponential is 'the rate of change is proportional to the amount'. Written symbolically, this is $y'(t) = ky(t)$.
			\item Recall (or verify) the critcial fact: if $y'(t) = ky(t)$ then $y(t) = Ae^{kt}$.
			\item It drives me (and most mathematicians) nuts when people misuse the word 'exponential' to mean 'getting bigger'.
			\item The concept of an exponential of a matrix is mindblowing, but also actually pretty easy. It can even be calculated without diagnalizing in the special case of a 'nilpotent' matrix. A matrix $A$ is called 'nilpotent' if there exists an integer $n$ such that $A^n =0$. You've encountered matricies like this before. A crowd favorite for this is in problem 11 and another is in 19 don't get fooled by this pattern, there are nilpotent matricies with entirely nonzero entries. Can you find one?
			\item Symmetry is an incredibly powerful tool, because diagnalization is even easier.
			\item Symmetric situations arise commonly in applications, so although it seems very niche, it's a pretty common scenario.
			\item Some of the problems in this section are frustratingly simple, but hard to see.
			\item Problems 12 and 24 are pretty tricky, don't feel badly if they are hard for you.
		\end{itemize}

	\subsection{Work}
		\begin{itemize}
			\item In section 6.3, please do problems 1-4, 8, 11-13, 21, 26, 31a, 32.
			\item In section 6.4, please do problems 4, 6, 7, 10, 11, 12, 15, 20, 23, 24, 34.
		\end{itemize}

	\clearpage

% <------------Differential Equations and Symmetric Matricies Ends Here------------>

% <------------Linear Transformations Begins Here------------>

	\section{Week 14 - Linear Transformations}

	\subsection{Readings}
		Please read sections 8.1 and 8.2.

	\subsection{Words of Advice}
		\begin{itemize}
			\item It may seem obvious that matricies and linear transformations are linked, but that's not a given. Be careful to use only facts from the appropriate area.
			\item You've been dealing with a great many linear transformations over the semester, so you should be getting pretty good at these kinds of things.
			\item In section 8.2, we'll see that linear transformations can be represented with a matrix.
			\item It should be apparent that $T(v) = Av$ is a linear transformation, some thoughts about this relationship are in problem 8.1 11. You might want to start 11 by showing that $T$ is a linear transformation.
			\item Section 8.2 is the punchline for this course. It shows that Matricies are Linear Transformations and Linear Transformations are Matricies.
			\item This means that everything we know about Matricies applies to Linear Transformations.
			\item We now know alot about Matricies, but there are three big takeaways that incorporate almost all of the material. Those are the Big Picture, Determinants, and Eigenpairs.
			\item In this last section, we also gain a thought about a tool we've been using all along. Diagonalization is really just a change of basis, this is intuitive, but also fantastic.
		\end{itemize}

	\subsection{Work}
		\begin{itemize}
			\item In section 8.1, please do problems 1-6, 8, 10, 11, 13, 17, 18, 20ab, 29ac, .
			\item In section 8.2, please do problems 1-7, 9, 10, 13, 14, 20-22, 26, 27, 34.
		\end{itemize}

	\clearpage

% <------------Linear Transformations Ends Here------------>

% <------------Review for Final Exam Begins Here------------>

	\section{Week 15 - Review for Final Exam}

	\subsection{Words of Advice}
		\begin{itemize}
			\item Fully explain all of your answers to the test. There's no credit for solutions without explanation.
			\item Explanations should cite facts from the book by page number and location. For example, 'top of page 123' or 'eqn (2) on page 123'.
			\item Justification should be provided for each step.
			\item You get credit for sound mathematical reasoning and careful explanation.
			\item Your explanations should be aimed at your peers level, not at me or Prof. Strang.
			\item Tests are assumed to be open book and notes, but are individual assessments.
			\item Please don't collaborate on tests.
			\item Excellent work making it to the end of the course. I can't imagine it's been an easy journey.
			\item Thank you for putting in the effort for this class.
			\item If you'd like something new to learn instead of reviewing, sections 6.5, 8.3, 10.4 and 10.5 all were close contenders for inclusion. 10.5 is probably the coolest of them.
			\item As with the previous exams, clear explanations are the only thing worth points on the final.
			\item Good Luck!
		\end{itemize}

	\clearpage

% <------------Review for Final Exam Ends Here------------>

